\newtheorem{lemma}{Lemma}
\newtheorem{theorem}{Theorem}
\newtheorem{cor}{Corollary}
\newtheorem{prop}{Proposition}
\newtheorem{Definition}{Definition}
\newtheorem{Proof}{Proof}
\newtheorem{property}{Property}
\newtheorem{subproblem}{Sub-Problem}
\newtheorem{examp}{Example}
%\newtheorem{algorithm}{Algorithm}
\newtheorem{Rule}{Rule}
\newtheorem{condition}{Condition}


\newcommand{\Hybrid}[1]{{\small{\textsf{Hybrid}}}}
\newcommand{\ANGIE}[1]{{\small{\textsf{ANGIE}}}}
\newcommand{\THOMAS}[1]{{\small{\textsf{RDF-3X}}}}
\newcommand{\YAGO}[1]{{\small{\textsf{YAGO}}}}


\newcommand{\blind}[1]{}

\newcommand{\Ex}{{\em \textbf{Example. }}}
\newcommand{\s}[1]{{\small{\textsf{#1}}}}
\newcommand{\SQ}[1]{{\small{\textsf{#1}}}}
\newcommand{\W}[1]{{\em {#1}}}





\newcommand\alkis[1]{{\color{blue} $\parallel$\textbf{Alkis}: #1 $\parallel$}}
\newcommand\serge[1]{{\color{red} $\parallel$\textbf{Serge}: #1 $\parallel$}}
\newcommand\nico[1]{{\color{blue}    #1   }}



\newcommand\calD{\ensuremath{\mathcal D}}
\newcommand\calB{\ensuremath{\mathcal B}}
\newcommand\calP{\ensuremath{\mathcal P}}
\newcommand\calI{\ensuremath{\mathcal I}}
\newcommand\calL{\ensuremath{\mathit{Label}}}
\newcommand{\cut}{\ensuremath{{\it cut}}}
\newcommand\locatepeers{{\tt locateXML}}

%\newcommand\ABF{\ensuremath{BF_{\alpha}}}
\newcommand\ABF{{\it ABF}}
%\newcommand\DBF{\ensuremath{BF_{\delta}}}
\newcommand\DBF{{\it DBF}}
\newcommand\anc{\ensuremath{ \backslash\backslash}}
\newcommand\stitle[1]{{\vpar\noindent{\bf #1}}}

\newcommand\vpar{{\vspace*{0.5em}}}

\newcommand{\csbbox}{\vrule height7pt width4pt depth1pt}

\newcommand{\fp}{\mathit{fp}}

\newtheorem{defn}{Definition}[section]
\newtheorem{thm}{Theorem}[section]
\newtheorem{lem}{Lemma}[section]

\newtheorem{CSEXAMPLE}{Example}[section]
\newenvironment{csexample}{\begin{CSEXAMPLE} \hspace{-.85em} {\bf :} \rm}%
                            {\end{CSEXAMPLE}}

\newenvironment{remark}{\stitle{Remark.}}{}

\newcommand{\csxam}{\begin{csexample}}
\newcommand{\csexam}{\csbbox\end{csexample}}

\newcommand{\csprf}{\noindent{\bf Proof:}}
\newcommand{\cseprf}{\csbbox}

\newcounter{ccc}
\newcommand{\bcc}{\setcounter{ccc}{1}\theccc.}
\newcommand{\icc}{\addtocounter{ccc}{1}\theccc.}

\newcommand\utitle[1]{{\vpar\noindent{\underline{#1}}}}
\newcommand\sutitle[1]{{\vpar\noindent{\textbf{\underline{#1}}}}}
\newcommand\bstitle[1]{{\noindent{\bf #1}}}

\newcommand\queryplan[3][@R=8pt @C=2pt]{{\small \begin{minipage}[th]{#2}\xymatrix
    #1 {#3}\end{minipage}}}
\newcommand\querytree\queryplan

\newcommand\notes[1]{{\bf $\|$~{#1}~$\|$}}

\newcommand\given{\ensuremath{~|~}}

\newcommand\eat[1]{}

\newcommand\minarg{\ensuremath{\operatornamewithlimits{minarg}}}
\newenvironment{pseudocode}{\begin{tabbing}llllll\=llll\=llll\=llll\=llll\=llll\kill\small}{\end{tabbing}}

\newcommand{\myhrule}{\rule[.5pt]{\hsize}{.5pt}}
\newcommand\smalltt[1]{\textsf{\small{#1}}}
%% Section spacing
%\newcommand{\oespace}{\vspace*{-1.8em}}
%\newcommand{\osspace}{\vspace*{-1.6em}}
%\newcommand{\ofspace}{\vspace*{-1.4em}}
%\newcommand{\otspace}{\vspace*{-1.2em}}
%\newcommand{\ospace}{\vspace*{-1em}}
%\newcommand{\espace}{\vspace*{-2mm}}
%\newcommand{\sspace}{\vspace*{-.6em}}
%\newcommand{\fspace}{\vspace*{-.4em}}
%\newcommand{\tspace}{\vspace*{-.2em}}


\newcommand{\pfspace}{\vspace*{.4em}}
\newcommand{\ptspace}{\vspace*{.2em}}
\newcommand{\paraspace}{\pfspace}

%%%%%%%%%%%% Fabian's macros

%\renewcommand{\algorithmicrequire}{\textbf{Input:}}
%\renewcommand{\algorithmiccomment}[1]{$~~~~$ //#1}
\newcommand{\ignore}[1]{}
% Fabian: I think we should use a uniform appearance of all entities
% in the text. It has proven useful to treat relations, entities and facts
% alike. Feel free to change the layout as you like it.
\newcommand{\entity}[1]{{\textit{#1}}}
%\newcommand{\var}[1]{{\textit{\$#1}}}
\newcommand{\newWord}[1]{\emph{#1}}

\newcounter{figureCounter}
\newcommand{\ffigure}[4]{% \ffigure[th]{label}{title}{imagecommand}{vskip}
\ \\[-0.0cm]
  \refstepcounter{figureCounter} \label{#1}
  {\centering #3\\[-#4]}
  {\centering \bf Figure \ref{#1}: #2\\[0.2cm]}
}
\newcommand{\para}[1]{\noindent{\textbf{#1.}}}


\newcommand{\hls}{{\small $<$}}
\newcommand{\hg}{{\small $>$}}
